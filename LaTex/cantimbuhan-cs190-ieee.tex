\documentclass[journal]{./IEEE/IEEEtran}
\usepackage{cite,graphicx}
\usepackage[T1]{fontenc}

\newcommand{\SPTITLE}{Reconstructing 3D Models from Images Using OpenMVG and OpenMVS}
\newcommand{\ADVISEE}{Mary Cris Joy Cantimbuhan}
\newcommand{\ADVISER}{Maureen Lyndel Lauron}

\newcommand{\BSCS}{Bachelor of Science in Computer Science}
\newcommand{\ICS}{Institute of Computer Science}
\newcommand{\UPLB}{University of the Philippines Los Ba\~{n}os}
% \newcommand{\REMARK}{\thanks{Presented to the Faculty of the \ICS, \UPLB\
%                              in partial fulfillment of the requirements
%                              for the Degree of \BSCS}}

\markboth{CMSC 190 Special Problem, \ICS}{}
\title{\SPTITLE}
\author{\ADVISEE~and~\ADVISER%
% \REMARK
}
\pubid{\copyright~2006~ICS \UPLB}

%%%%%%%%%%%%%%%%%%%%%%%%%%%%%%%%%%%%%%%%%%%%%%%%%%%%%%%%%%%%%%%%%%%%%%%%%%

\begin{document}

%TITLE
\maketitle

% \begin{abstract}
% Your abstract.
% \end{abstract}

\section{Introduction}

\subsection{Background of the Study}
Back in the 1960s, computer engineers have been creating three-dimensional (3D) models. In our current age, it is used in many things such as animation, games, visualization and 3D printing \cite{Archicgi}.

Using multiple images in reconstructing 3D models has been around for decades. There are many approaches on this, such as \textit{snakes, silhouette exploitation} and \textit{patchwork} \cite{Paris}. All of these are still relevant and some of them are used together with other techniques.

\textit{Structure from Motion (SfM)} is a technique to obtain camera parameters for 3D model reconstruction. It has been used since 1999 \cite{Jebaraa}, until today. OpenMVG uses this as its basis \cite{OpenMVG}.

The study will focus on SfM-MVS (Structure from Motion-Multi-View Stereo). Where SfM by OpenMVG will be used in reconstructing a basic 3D point cloud, and OpenMVS will be used in creating mesh and applying texture \cite{OpenMVS}.

\subsection{Statement of the Problem}
Statement

\subsection{Objectives}
Generally, the study aims to develop an application that will accept images from users, and then produce a 3D model from those images. Specifically, this study aims to:

\begin{itemize}
\item To develop an application for 3D reconstruction from images
\item To use SfM-MVS workflow in creating 3D models
\item To assess the application\'s ability to reconstruct 3D models from images within the limitations
\end{itemize}

\subsection{Significance of the Study}
why?

\subsection{Scope and Limitations}
grayscale, smooth, dark, machine(?)
where, when
% Your introduction goes here! Some examples of commonly used commands and features are listed below, to help you get started. If you have a question, please use the help menu (``?'') on the top bar to search for help or ask us a question. 

\section{Literature Review}
intro to this chapter

\subsection{Related Studies}
chop some stuff

\subsection{Definition of Terms}

\section{Hypotheses}
null ad alternative hypotheses

%MATERALS AND METHODS
\section{Materials and Methods}
intro to this section

\subsection{Materials}
laptop, camera, programs, subjects

\subsection{Data Collection}
try uli kung meron sa net. visit places like Paete to take pictures

\subsection{Methods}
numbered nerdy stuff. add figures, formula, flow chart, algo

\subsection{Data Analysis}
how can we consider this successful/unsuccessful? holes?

\section{Nature and Form of Results}
research paper. and an app

%PLAN
\section{Budget and Project Timeline}
intro to this section

\subsection{Project Budget}
table: breakdown

\subsection{Project Timeline}
gantt chart of activities (per week)

% APPENDICES
\appendices

\section{Proof of the First Zonklar Equation}
Appendix one text goes here...

\section{}
Appendix two (without title) text goes here...

% ACKNOWLEDGMENT
\section*{Acknowledgment}
Many thanks to... the quik brown fox jumps over the lazy dog.

% BIBLIOGRAPHY
\bibliographystyle{./IEEE/IEEEtran}
\bibliography{./cantimbuhan-cs190-ieee}
% \nocite{*}

\end{document}